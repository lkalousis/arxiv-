\documentclass[a4paper,11pt]{article}
\pdfoutput=1 

%\usepackage{jinstpub} 
%

\usepackage{enumitem}
 
\usepackage{amssymb}
\usepackage{amsmath}
%

\begin{document}

{\bf Answers to referee \#1}
\\[1ex] 

I would like to thank the referee for his or her nice comments.
I was flattered to see that, he or she, took the time to read my work carefully and understood the main takeaway messages.  
My answers below:
\\[1ex]

C: \emph{ One small remark in this section concerns figure 1: I suppose that the values of mu reported in the figures are the ones from the fit and that the ones set in the toy montecarlo are 2 and 3 respectively; 
for clarity, this should be explicitly mentioned in the caption of the figure. }

A: Done. The caption of the figure now reads:
\\[1ex]

A few fits obtained with the analytical model. 
Black dots show the generated data and the azure line shows the best fit. The dashed lines show the contributions from the several PE peaks. 
The spectra generated with true values of $\mu$ equal to two (top) and three (bottom) are shown.
\\[1ex]

C: \emph{In this section I would only suggest the authors to anticipate the unit of the gain and sigma to the caption of table 1 instead of letting the reader dedice them from the X-axis legend of figure 4.}

A: That was an important omission. I have added the following sentences in the text :
\\[1ex]

Figure~4 shows  a few fits obtained with the $S_R(x)$ model put forward in this article. 
Note that the unit of the $x$ axis is nanovolt times second (nVs), which is the unit of charge given by the LeCroy oscilloscope.
In particular, $Q$ and $\sigma$ are measured in nVs and $\alpha$ in nVs$^{-1}$.
\\[1ex]

I  added  also the units of $\alpha$, $Q$, $\sigma$ and $Q_s$ in tables 1 and 2. 
\\[1ex]

C: \emph{ I would also suggest to mention the fitting error on the gain in the caption of Table 2: 
this will give a more quantitative argument to appreciate the similarity of the analytical and DFT results and the negligible drift effect on the gain. }

A: Good point. The errors on the gain have been calculated and added in table 2.
One sees that the deviation is of the level of 2$\sigma$; well within the errors. 
A comment has been added:
\\[1ex]

We should also note that a drift in the gain is observed (owing to the correlations highlighted in previous section) but this is too small in magnitude.
It is a deviation of $\sim$ 2$\sigma$ and it is well within the measurement errors.    
\\[1ex]

C: \emph{ In the first sentence, it is mentioned that the proposed analytical method can be applied to PMT's "whenever the SPE response can be parameterized by eq. (4)". 
Can the autohrs be more explicit on the properties of the PMT's where this method can be used? 
Or in other words, can one have an expample of PMT's whre this method woudld fail and why? }

A: I have added the following lines in the text:
\\[1ex]

In this article we have presented an analytical model that can be used for the calibration of PMTs whenever the SPE response can be parameterized by eq.~5.  
Note that the SPE charge amplification of the full dynode chain for the R7081 PMT has a very symmetrical shape and it can be modeled by a truncated gaussian.  
In several cases, as for instance in the R1408 Hamamatsu PMT, the SPE response has a rather asymmetrical shape and the formulae given in this publication seize  to apply. %\footnote
In this example a gamma distribution is to be preferred~[4]. 
\\[1ex]


\vspace{1cm}
{\bf Answers to referee \#2}
\\[1ex]

I would like to thank the referee for his/her valid comments. 
I have tried to address them in the lines that follow: 
\\[1ex]

C: \emph{I feel the contents of the paper interesting, but more suitable for a technical note than a full research paper, as it just regards the details of the calculations needed in methods illustrated elsewhere.}

A: I can see the referee's point. Indeed, the article is quite technical, with a lot of mathematical formulae and it is intended for the narrow field of PMT calibration and monitoring. 

Nonetheless, I must equally mention that the Dossi model has been published in the distant 2000 and since then no analytical solution has been offered. 
At least, no other solution with the same level of accuracy and precision. 
As a quick example, I can refer to the DarkSide paper:
\\[1ex]

DarkSide collaboration, Light Yield in DarkSide-10: A Prototype Two-Phase Argon TPC for Dark Matter Searches, Astropart. Phys. 49 (2013) 44 [arXiv:1204.6218].
\\[1ex]

\noindent where the Dossi model is employed. 
The authors solved the equations numerically for the first two PE peaks and higher peaks were approximated with symmetric gausssians. 
And even though I am not acquainted with the fine details of the DarkSide paper, from my work I can tell that such an approach is bound to fail
for large values of $\mu$. Not to mention the running time that is needed to compute the various convolutions numerically.  

In striking contrast, the model that I have presented can incorporate higher PE corrections with only a few lines of code. 
A solution exists in a closed form and the software can run with only a few seconds of execution time. 
I personally believe that this is a significant improvement over the existing approximate methods and the paper is worthy of a NIMA publication.  

I have added the following lines in the text (Section 5, Outlook) to highlight these points:
\\[1ex]

Finally, we should also mention that the machinery developed in this article comprises the first analytical solution of the Dossi model. 
As an example, we can refer to the Darkside paper, ref.~[9], where the equations were solved numerically for the first two PE peaks 
and higher peaks were approximated with symmetric gaussians. 
This approach is accurate only for small values of $\mu$, where the non-gaussian tails of $S^{(n)}(x)$ can be neglected. 
In striking contrast, the model of this article can incorporate higher PE corrections, in the software, with only a few lines of code. 
Additionally, a solution exists in a closed form and a single spectrum can be analyzed in just a few seconds of execution time. 
We believe that this is an important improvement over the existing approximate methods. 
\\[1ex]

C: \emph{ line 14-19 please add some explanations for S(x) and define the various used items, such as $g_N$, Q, H(x) ... }

A: The lines of text have been added :
\\[1ex]
Years later, R. Dossi \emph{et al.} proposed a specific model for the single photoelectron (SPE) response function $S(x)$ that can treat a vast number of PMTs~[2]. 
It was based on a carefully chosen combination of an exponential and a truncated gaussian distribution:
\begin{align}
S(x) =    \left( \ w \alpha e^{-\alpha x } + \frac{(1-w)}{g_N} \frac{1}{\sqrt{2\pi}\sigma} e^{ - \frac{( x - Q )^2}{2\sigma^2}} \ \right) H(x).       \label{eq:S}
\end{align}
Note that photoelectrons that backscatter and miss the first amplification stage are modeled in $S(x)$ through the exponential term. 
The slope of the exponential is $\alpha$ and the $w$ pre-factor parametrizes the probability for this process to happen. 
A better discussion of this, with further experimental tests, can be found in ref.~[2]. 
$H(x)$ is the Heaviside step function and $Q$ and $\sigma$ are the parameters of the truncated gaussian that models the amplification of the full dynode chain.  
The $g_N$ factor ensures that the truncated gaussian is properly normalized and equals to:
\begin{align}
g_N = \frac{1}{2} \textrm{erfc} \left( - \frac{Q}{\sqrt{2}\sigma} \right), 
\end{align}
where erfc($z$) is the complementary error function~[3]. 
\\[1ex]

C:\emph{line 48... please define $S_R(x)$ ...}

A: The lines of the text have been modified to make this clearer:
\\[1ex]

The behavior of a PMT, when illuminated with light pulses of constant number of photons, is dictated by the charge response function $S_R(x)$:
\\[1ex]

C: line 73 recieves $->$ receives

A: Done
\\[1ex]

C: line 96 lied $->$ was

A: Done
\\[1ex]

C: \emph{line 162-189 please put a figure illustrating the used setup}

A: Done 
\\[1ex]

C1: \emph{Table 1 how these results compare with the other outlined calibration methods ?}

C2: \emph{Table 2 what will be the results instead with method of reference [2]?}

A: I took the liberty to answer both referee's comments with a single answer. 
Mainly because they fall in the \emph{Comparison} section of the article. Kill two birds with the same stone. 
So, all three methods should return the same results of tables 1 and 2. 
A comparison of the DFT and numerical integration methods is presented in ref.~5. 
The only thing is that the numerical integration gives a slightly worse  $\chi^2$/NDOF and took more time to analyze the data.  
 
I have added the paragraph in subsection 4.2 :
\\[1ex]

Furthermore, in ref.~5 we have analyzed the same data set using the numerical integration method mentioned in section~1.  
To simplify the calculations only the first PE peak was worked out numerically and higher peaks were approximated with perfectly symmetric gaussians. 
It was shown that inside the $\mu$ = 0.5 -- 2.0 window the numerical integration gave the same results as the DFT approach. 
So, we can expect all three techniques to provide the same figures of tables 1 and 2. 
Additionally, it should be mentioned that the numerical integration gave a slightly worse $\chi^2$/NDOF and took more time to analyze this data set. 
Calculating more peaks numerical will improve $\chi^2$/NDOF but at the same time will increase the execution time rendering the method almost inapplicable. 
More details can be found in ref.~5. 
\\[1ex]

C2: \emph{fig 3 please add chi2/NDOF as in previous fig1}

A: Good point ! Done !

\end{document}
