\documentclass[a4paper,11pt]{article}
\pdfoutput=1 

%\usepackage{jinstpub} 
%

\usepackage{enumitem} 
%

\begin{document}

{\bf Answers to the referee}
\\[1ex]

I would like to thank the referee for all of his comments. 
I have tried to address them in the lines that follow: 
\\[1ex]

C: \emph{Unfortunately the paper does not touch one of the main problems of the PDE problem -- photoelectron backscattering. It would be reasonable to discuss it a bit at least.}

A: Photoelectrons that backscatter and miss the first dynode are treated in the Dossi \emph{et al.} paper through the exponential term included in the SPE response function.  
The probability for this process to happen is dictated by the $w$ pre-factor. A better discussion of this can be found in the Dossi paper together with more information.

I have added the following lines in the text:
\\[1ex]

Note that photoelectrons that backscatter and miss the first amplification stage are modeled in the SPE function through the exponential $f(x)$ term. 
The $w$ pre-factor parametrizes the probability for this process to occur. A better discussion of this, with further experimental tests, can be found in ref.~[2]. 
\\[1ex]

C: \emph{Comparisons with experimental data are scarce.}

A: It all depends on what you want to do. In this article, all I wanted to accomplish was:
\begin{enumerate}
\item Work out the mathematics of the model in a clear way that leaves no doubt about the validity of the final result and,
\item Show that this function can be used for gain determination, and that it returns the same values with the existing numerical techniques.
\end{enumerate}
I believe that these points were were demonstrated. 
Of course, further data could be used to study other properties of the Hamamatsu R7081, but that was outside the scope of the current publication. 

To reflect these views, I have added the following text in the Outlook:
\\[1ex]

Last, we would like to emphasize that the main thrust of our efforts was directed towards two specific objectives:
\begin{enumerate}
\item First, we wanted to derive all the necessary mathematical formulae in a clear and pedagogical manner that everybody understands and,
\item Second, we wanted to demonstrate that the model can extract the gain parameter in a rigorous way, similar to those of already established methods. 
In this respect, we showed that it returns figures consistent with the numerical DFT technique.  
\end{enumerate}
Of course, the procedure outlined in this publication can be used to study further properties of PMTs with gaussian single photoelectron response function. 
For instance, it could be employed to probe the photoelectron detection efficiency (PDE) of PMTs with characteristics similar to the Hamamatsu R7081. 

C: \emph{It would be good to add comparisons with responses of photomultipliers of other types. }

A: Unfortunately this cannot be done. 
The Hamamatsu R7081 model was the only PMT with a gaussian SPE charge response that we had. 
Note also, that our setup has been dismantled and taking more data now is impossible.  
\\[1ex]

C: \emph{The paper's language is rather sloppy}

A: I was equally stunned and surprised to read this line from the referee. 
In my opinion, I have made the effort to write this article in scientific, yet pedagogical way, that meets the standards of a JINST publication. 
I would like to ask the referee to direct me to specific places in text that, he or she, believes that require revision. 
\\[1ex]

C: \emph{Many important references appropriate for the paper are missed.}

A: I have tried to include the minimal amount of references necessary to go through the material covered in this publication. 
The Bellamy paper, the Dossi \emph{et al.} model, etc ... Mastering these articles one can follow and understand the paper without major problems. 
I am not sure if any further citation is required. 
Nonetheless, I would like to ask the referee to point out specific places in the text where, he or she, believes that a reference has been omitted, or that a citation is deemed absolutely necessary.  
\\[1ex]

C: \emph{On the other hand there are references to wikipedia which should certainly be replaced. }

A: All wikipedia references have been replaced with a citation pointing to the standard Abramowitz and Stegun textbook:
\\[1ex]

M. Abramowitz and I. A. Stegun, \emph{Handbook of Mathematical Functions: with Formulas, Graphs, and Mathematical Tables}, Dover Publications, 0009-Revised edition (June 1, 1965).  

\end{document}
