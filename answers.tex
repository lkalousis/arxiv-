\documentclass[a4paper,11pt]{article}
\pdfoutput=1 

%\usepackage{jinstpub} 
%

\usepackage{enumitem} 
%

\begin{document}

{\bf Answers to the referee}
\\[1ex]

I would like to thank the referee for his comments. 
I have tried to address them in the text that follows: 
\\[1ex]

C: \emph{Unfortunately the paper does not touch one of the main problems of the PDE problem -- photoelectron backscattering. 
It would be reasonable to discuss it a bit at least.}

A:
\\[1ex]

C: \emph{Comparisons with experimental data are scarce.}

A:
\\[1ex]

C: \emph{It would be good to add comparisons with responses of photomultipliers of other types. }

A: Unfortunately this cannot be done. 
The Hamamatsu R7081 model was the only PMT with a gaussian SPE charge response that we had. 
Note also, that our setup has been dismantled and taking more data is now impossible.  
\\[1ex]

C: \emph{The paper's language is rather sloppy}

A: I was equally stunned and surprised to read this line from the referee. 
In my opinion, I have made the effort to write this article in scientific, yet pedagogical way, that meets the standards of a JINST publication. 
I would like to ask the referee to direct me to specific places in text that, he or she, believes that require revision. 
\\[1ex]

C: \emph{Many important references appropriate for the paper are missed.}

A: I have tried to include the minimal amount of references necessary to go through the material covered in this publication. 
The Bellamy paper, the Dossi \emph{et al.} model, etc ... Mastering these articles one can follow and understand the paper without major problems. 
I am not sure if any further citation is required. 
Nonetheless, I would like to ask the referee to point out specific places in the text where, he or she, believes that a reference has been omitted, or that a citation is deemed absolutely necessary.  
\\[1ex]

C: \emph{On the other hand there are references to wikipedia which should certainly be replaced. }

A: All wikipedia references have been replaced with a citation pointing to the standard Abramowitz and Stegun textbook:

M. Abramowitz and I. A. Stegun, \emph{Handbook of Mathematical Functions: with Formulas, Graphs, and Mathematical Tables}, Dover Publications, 0009-Revised edition (June 1, 1965).  


\end{document}
